% !TEX root = ../main.tex
\section{Discussion}
\label{sec:discussion}
\subsection{Perceptual Uncertainty}
An important problem is to formally define the perceptual uncertainty produced by any detection algorithm similar to those in SLAM. For example, if the detector knows the pose estimated have large margins of error, it is useful to have some notion of uncertainty around the estimated pose. These are especially helpful inputs to any motion planing algorithms that takes uncertainty into consideration. In our implementation, we have a loose definition of detection uncertainty. We implicitly treated the pose estimation from depth sensor and RGB sensor as two independent observations. Uncertainty of the pose can be described by the variance and differences between two observations. This can be achieved using a similar technique to the updating step in Kalman filters.

\subsection{Computation Time}
We briefly tested the computation time of the new algorithm. With our current implmentation in Python, the algorithm can process a $pixel by pixel$ image in ~$xxx$ seconds. All tag detectors and the fusing process were running in a single-threaded mode of an Intel core. Since our sensory updates at roughly $35Hz$, the entire pipeline can process the tags in real time. There is no signifciant time increase on a higher resolution image because our fusing algorithm does not need to process the entire image. Therefore, the only time increase comes from the initial Apriltag detection process which has been shown to work under $30$ ms for large images.
\scnote{These are results. They should not be in discussions. Also in general you may want to reorganize some of the results. Essentially you're saying that there are 3 sources of difficulty - perspective, distance and illumination and you want to model the noise due to each and also compare the normal and fused methods w.r.t each of these factors, and finally show that the computation time of the fused is competitive with the normal. There has to be an easier way to represent all this. }

The most time consuming step is running the trust region optimization for refining the pose. This process can be sped up significantly by simply implementing the pipeline in C++.