% !TEX root = ../main.tex
\section{Introduction}
\label{sec:intro}
Detection and identification using artificial landmarks has long been popular in augmented reality and computer vision. Planar markers such as ARTag, ARToolkil, are among the most popular forms of fiducial markers. Compared to markerless detection algorithms, fiducial tags are much more reliable. High detection rates and accurate encodings make them popular in robotic applications such as testing SLAM systems, finding ground truth for manipulation tasks.    

In robotic applications, it is important to be able to accurately and reliably calculate the pose of the fiducial tags. There has been large amounts of effort for improving the detection algorithm by making them faster and more accurate using the RGB images. These algorithm yield great results under well conditioned or simulated environments. (Provide reference to specific studies). However, in real robotic application, these algorithms are often tested under various lighting and sensory noises. In these conditions, the fiducial tags suffers greatly from the perceptual ambiguity problem and makes the pose estimation difficult without additional information. In fact, we observe that the localization accuracy of the state of the art Apriltags is significantly worse at difficult viewing angles or when there are noise in the scene. 

In this paper, we present an algorithm that take advantage of the RGBD sensor to accurately estimate the pose from a single tag under noisy conditions. There are few key features to this algorithm: 
\begin{itemize}
\item The algorithm is generalizable to all square based fiducial tags.
\item The algorithm performs at worse as good as only using RGB images.
\end{itemize}